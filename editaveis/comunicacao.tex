\section{Plano de gerenciamento de comunicações}

\subsection{Objetivo}

  O objetivo deste é documentar as formas de comunicação realizada entre os integrantes do projeto Agricultura Urbana FGA, esclarecendo e
  assegurando a forma de troca de informações.

\subsection{Introdução}

  “A comunicação nas organizações é uma das formas mais eficazes da corporação sobreviver e prosperar no mercado competitivo atual.
  Essa ferramenta estabelece um relacionamento entre a empresa e os funcionários, permitindo que eles acompanhem as suas principais
  ações e verifiquem o impacto de suas tarefas no alcance dos resultados traçados. Com isso, os colaboradores ficam mais motivados a
  contribuir para o sucesso da companhia.” Marques, José Roberto. Presidente da IBC

  Um grande problema do gerenciamento e produção de qualquer projeto é a comunicação entre os integrantes. Em um projeto grande, é muito
  importante definir o escopo e se assegurar que todo o time possui a mesma ideia sobre como seguirá.

  Para este foi levado em conta o tamanho da equipe e a disponibilidade de horários, além das ferramentas que são disponíveis para todos.

\subsection{Ferramentas utilizadas}

  Foram utilizadas apenas ferramentas cruciais para o desenvolvimento do projeto. Evitando assim que o excesso de ferramentas dificulte
  a comunicação.

  Assim, as seguintes ferramentas são apresentadas:

  \begin{table}[!htb]
    \centering
    \begin{tabular}{p{2,5cm}p{5cm}p{7cm}}
      \toprule
        \textbf{Ferramenta} & \textbf{Descrição} & \textbf{Finalidade} \\
      \midrule
        Telegram & Ferramenta de comunicação geral da equipe & Avisos rápidos, votações, marcar reuniões e tirar dúvidas \\ \midrule
        Google Hangouts & Ferramenta de videoconferência & Realização de reuniões não presenciais \\
      \bottomrule
    \end{tabular}
    \caption{Ferramentas de comunicação}
  \end{table}

\subsection{Estratégia de comunicação}

  A comunicação do grupo por completo será feita pelo Telegram, contendo todos os 35 integrantes, centralizando as informações principais
  sobre o projeto. O Telegram nos possibilita dividir em grupos menores, no caso são cinco sub-grupos de acordo com os principais temas.
  Assim,  todos assuntos poderão ser tratados apenas em grupos a onde são pertinentes.

  O grupos que utilizados no telegram são: PI $-$ Agricultura FGA, Logística, Infraestrutura, Automação, Nutrição e Legislação.

  Desta forma há forte interação entre os integrantes dos grupos e entre os grupo.

\subsection{Comunicação interna}

  A tabela que segue mostra as atividades de reunião e comunicação que são necessárias para a gestão e desenvolvimento do projeto.

  \begin{table}[!htb]
    \centering
    \begin{tabular}{p{3cm}p{5cm}p{3cm}p{1,5cm}p{3cm}}
      \toprule
        \textbf{Atividade} & \textbf{Objetivo} & \textbf{Frequência} & \textbf{Horário} & \textbf{Ferramenta ou local} \\
      \midrule
        Reunião em aula                         & Realização do desenvolvimento do projeto com todos os membros presentes.      & 2x/semana         & 14 ás 16  & Auditório, FGA    \\ \midrule
        Avisos Rápidos                          & Notificar mudanças ou avisos.                                                 & Quando necessário & N/A       & Telegram          \\ \midrule
        Reuniões dos integrantes dos subgrupos  & Realização do desenvolvimento do projeto com todos os membros dos subgrupos.  & Quando necessário & N/A       & Telegram, FGA     \\ \midrule
        Reunião da subgerência                  & Gerenciar andamento de tarefas do projeto.                                    & Quando necessário & N/A       & Telegram, FGA     \\ \midrule
        Reuniões extraordinárias                & Resolver problemas diversos.                                                  & Quando necessário & N/A       & Telegram Hangouts \\
      \bottomrule
    \end{tabular}
    \caption{Comunicação interna}
  \end{table}

