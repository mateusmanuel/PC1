\chapter{Logística do projeto}

  Os artefatos da parte de logística, que é responsável pela formação estratégica do projeto, serão retirados a partir do guia PMBOK \cite{pmbok}
  ou Guide to the Project Management Body of Knowledge, que é o guia do conjunto de conhecimento de gerenciamento de projeto na qual
  inclui: Plano de gerenciamento de Escopo, projeto, recursos humanos, custo, tempo, comunicação e riscos;

\section{Plano de gerenciamento de recursos humanos}

  O plano de gerenciamento de recursos humanos visa estabelecer a gestão das equipes durante a realização de todo o projeto de modo
  a atingir os objetivos do projeto.

\subsection{Origem dos recursos}

  O projeto Agricultura Urbana FGA é formado por trinta e quatro (34) integrantes, distribuídos por cinco equipes: nutrição, logística,
  infraestrutura, legislação e automação, para o primeiro ponto de controle teremos maior ênfase na logística para o planejamento do
  projeto, na qual todas as equipes estão colaborando com tal planejamento na medida do possível, de modo que se for necessário, as equipes serão redistribuídas em cada ponto de controle para atender a demanda do projeto. No segundo ponto de controle
  terá uma ênfase maior na parte de automação devido a fase inicial de criação do protótipo do projeto, sendo necessário realocar pessoas que sejam capacitadas a realizar essa tarefa. A equipe de nutrição, infraestrutura e legislação continuará a realizar pesquisas de modo mais aprofundado, e a equipe de logística irá fazer manutenções nos planos de gerenciamento e ajudar na medida do possível as outras. No terceiro ponto de controle teremos uma grande ênfase na equipe de logística para a criação do plano de gerenciamento de custos do projeto e na equipe de automação para a finalização do protótipo de alta fidelidade, demonstrando uma maquete do projeto e as telas do software que irá controlar todo os sistema.

  No primeiro ponto de controle a equipe de nutrição será formada por seis (6) integrantes, a equipe de infraestrutura conta com oito (8)
  integrantes, a equipe de Legislação com sete (7) integrantes, a equipe de automação não estará presente neste ponto de controle e se
  juntará a equipe de logística que ficará com o restante dos integrantes. No segundo ponto de controle iremos aumentar o número de
  integrantes na equipe de automação - para a criação do protótipo - e nas equipes de pesquisa e diminuir os integrantes da equipe de
  logística. No terceiro ponto de controle iremos diminuir as equipes de legislação, nutrição e infraestrutura para suprir as equipes de
  logística e automação, de modo a priorizar a finalização do protótipo e dos planos de custo, tendo alguns membros para terminar pesquisas e a equipe toda fazer a conclusão do projeto.

  Todos os membros do projeto são divididos entre graduandos dos cursos de Engenharia de Software da Faculdade do Gama, Engenharia Automotiva da Faculdade do Gama, Engenharia Aeroespacial da Faculdade do Gama, Engenharia de Energia da Faculdade do Gama, Engenharia Eletr\^{o}nica da Faculdade do Gama, todos na Universidade de Brasília.

\subsection{Equipe e Responsabilidade}

  As tabelas com todas as equipes do projeto se encontra no anexo \ref{equipes}

\subsection{Treinamento}

  A equipe de automação pode necessitar de treinamento para que consiga criar, operar e solucionar possíveis problemas com os equipamentos
  utilizados na área de plantio e para a criação do protótipo, além do treinamento para utilização eficaz da ferramenta trello.

\subsection{Acompanhamento da equipe}

  Para acompanhar e controlar a equipe são realizados encontros às segundas e quartas-feiras onde é exposto o que foi feito e o
  que falta ser concluído, além disso, foram alocados subgerentes para acompanhar e organizar as diversas áreas do projeto e um gerente
  que acompanha todo o processo, vale ressaltar que a equipe, por utilizar uma adaptação da metodologia ágil SCRUM para o projeto, se
  autogerencia e se auto-organiza de modo que facilita o trabalho dos gerentes para que eles possam, também, contribuir para o projeto.
