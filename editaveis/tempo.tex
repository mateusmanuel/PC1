\section{Plano de gerenciamento de tempo}

\subsection{Objetivo}

  Este documento tem como finalidade estabelecer como será o gerenciamento do tempo do projeto, mostrando como o cronograma foi criado e
  evoluído ao longo do projeto.

\subsection{Processo de gerenciamento de tempo}

  Seguindo o PMBOK 5ª edição, serão realizadas as seguintes etapas no processo de gerenciamento do tempo durante o projeto: Planejamento
  e gerenciamento do cronograma; Definir atividades; Sequenciar as atividades; Estimar os recursos das atividades; Estimar as durações
  das atividades; Desenvolver o cronograma e controlar o cronograma.

\subsection{Atividades}

  As atividades foram definidas com base nas entregas necessárias para o término bem sucedido do projeto, sendo que elas foram mapeadas posteriormente com a Estrutura Analítica do Projeto (EAP).

\subsection{Cronograma}

  A EAP e o cronograma do projeto estão anexados no \ref{eap}.

\subsection{Gerenciamento de mudanças}

  Para melhor gerenciamento das mudanças, foram definidos três níveis de impacto destas que podem vir a ocorrer no cronograma e as
  medidas a serem tomadas.

  \begin{table}[!htb]
    \centering
    \begin{tabular}{p{5cm}p{10cm}}
      \toprule
        \textbf{Impacto da mudança} & \textbf{Medida a ser tomada} \\
      \midrule
        Muito alto           & Reunião extraordinária da equipe de gerência no hangout para discussão das medidas a serem tomadas e
                               aviso das mudanças \\ \midrule
        Alto ou moderado     & Reunião da equipe de gerência para discussão e aviso das mudanças no telegram \\ \midrule
        Baixo ou muito baixo & Aviso das mudanças a serem realizadas para todas as equipes através do telegram \\
      \bottomrule
    \end{tabular}
    \caption{Níveis de impacto na mudança do cronograma}
  \end{table}

