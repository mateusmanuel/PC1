\chapter{Pesquisa e Desenvolvimento}

\section{Nutrição}

\subsection{Plantação utilizando hidroponia}

  Em um período não tão distante, diversas descobertas surgiram provando que há, verdadeiramente, um elo entre a nutrição apropriada e o rendimento
  acadêmico. Desta forma, os baixos níveis de proteínas, ferro e até mesmo a fome temporária foram listados como problemas decorrentes
  de nutrição inadequada, podendo ocasionar irritabilidade, fadiga, dificuldade de concentração entre outros \cite{nutricao1}.

  Dentro deste contexto, surge a necessidade de prover para a comunidade acadêmica a possibilidade de manter uma alimentação adequada,
  de maneira que se possa ter uma boa memória, um cérebro alerta e consecutivamente, vantagem no estudo \cite{nutricao8}.

  Por conseguinte, foi priorizado como um dos pilares deste projeto uma plantação de frutas, verduras e legumes feita utilizando hidroponia,
  técnica de cultivar plantas trocando o solo por uma solução aquosa preparada, visando a necessidade de cada espécie cultivada.
  Esse método foi escolhido por facilitar na hora de se monitorar o plantio, para se automatizar o sistema e para pesquisas que possam ser
  realizadas por projetos de extensão. Para melhores resultados no processo de cultivo será instalado um sistema que irá automatizar o
  processo de plantio, controle da solução aquosa (quantidade de nutrientes, ph da solução, etc) e colheita.

  Como maneira de dar aos estudantes o benefício da nutrição saudável, foram listados os seguintes vegetais:

  \begin{itemize}
    \item \textbf{Cenoura}: A cenoura é muito rica em betacaroteno (poderoso antioxidante e anticancerígeno e ajuda no desempenho dos
      receptores da retina, melhorando a visão), fonte de fibras, minerais (fósforo, potássio, cálcio e sódio) e vitaminas (A, B2, B3 e C) \cite{nutricao2}.
    \item \textbf{Morango}: O morango é fonte abundante de vitamina C, que é antioxidante e previne gripes, infecções e ainda fortalece os
      dentes e ossos. Possui antocianina (subst\^{a}ncia que dá sua cor) e auxilia na prevenção do envelhecimento precoce da pele e da
      arteriosclerose \cite{nutricao3}.
    \item \textbf{Folhas escuras}: As folhas verdes são ricas em nutrientes importantíssimos para a saúde. São fontes de fibras, vitaminas
      e minerais: destaque especial para o cálcio e o ácido fólico \cite{nutricao4, nutricao5}.
  \item \textbf{Tomate}: Esse fruto é rico em licopeno, o tomate está associado a índices reduzidos de c\^{a}ncer de p\^{a}creas,
      cervical e próstata. Ele protege o organismo de infecções bacterianas, assim como de perturbações digestivas e pulmonares \cite{nutricao5}.
    \item \textbf{Maracujá}: contém alcaloides e flavonoides, subst\^{a}ncia que agem no sistema nervoso central e atuam como tranquilizantes,
      analgésicos e relaxantes musculares. Por isso ajudam a combater a ansiedade, a depressão e os distúrbios do sono \cite{nutricao12}.
  \end{itemize}

\subsection{Plantação nos locais ociosos do MESP}

  No projeto inicial estava previsto que também seria um pilar a ocupação dos locais ociosos do prédio MESP, onde se localiza o
  Restaurante Universitário (RU). Seria um ponto ideal para produzirmos condimentos em uma área que hoje se encontra inutilizada.
  Dessa forma, apesar das pesquisas efetuadas terem mostrado esse local como inadequado - pois se trata de uma área sem irradiação solar - aparatos tecnológicos podem suprir tal necessidade, de modo que o plantio nesse lugar não se torna completamente descartado dos planos do projeto.

  Com isso, a saída encontrada para a ideia inicial de plantação de condimentos para a comunidade acadêmica, foi a utilização da área
  entre o prédio MESP e a Quadra Poliesportiva, por ser considerada um local de fácil acesso para os consumidores da mini horta
  comunitária e também por ser um local ocioso da Faculdade do Gama.

   Seria possível então, transformar aquelas áreas, que não oferecem nada a comunidade, em locais onde os próprios alunos pudessem
   usufruir de um bem saudável em sua própria alimentação.

   Os condimentos sugeridos foram escolhidos por terem pequenas ramificações e atenderem a comunidade acadêmica, sendo eles:
   Manjericão, Cheiro-Verde (salsa e cebolinha), Coentro, Tomilho e Pimenta.

\subsection{Pomar e área de convivência}

  O nosso terceiro pilar é baseado em um pomar, tendo em vista que uma grande demanda entre os alunos da FGA é a de espaços de convivência.
  A praça existente é exposta ao sol, o que impede que os alunos desfrutem do espaço. Com o plantio de árvores frutíferas, teríamos duas
  soluções para o lugar: Espaços de convivência para os estudantes poderem interagir, estudar e descansar e um aumento na qualidade de vida
  deles tanto pelo lazer quanto pelas opções de frutas frescas direto do pé.

  O pomar se dividirá em duas parte, entre o prédio de aulas e o MESP, onde seriam alocados bancos entre as árvores, e entre o prédio de
  aulas e o prédio de laboratórios, onde já existe a praça. O plantio das árvores frutíferas seria feito para sombrear a praça e refrescar,
  já  que a presença de  árvores aumentará a umidade.

  Foram indicadas as seguintes frutas para o pomar: araticum, cajuí, mangaba, jabuticaba, cagaita, pera-do-cerrado, cereja-do-cerrado,
  pitanga, amora e manga. A escolha das frutas foi feita com base na adequação delas ao solo presente na FGA, ao alto índice de
  vitaminas e a grande aceitação entre o público. Frutas vermelhas e roxas como a amora e a jabuticaba possuem antioxidantes que ajudam a
  proteger o cérebro e aumentam a capacidade cognitiva, a memória e o desempenho cerebral. As frutas escolhidas também são ricas em
  vitamina A, que também colabora para o processo de aprendizado \cite{nutricao7}.

  Com a proteção de árvores, um tempo ao ar livre ajuda a aumentar a concentração do aluno nas aulas, e a exposição moderada ao sol
  ajuda a fixar a vitamina D no organismo, melhorando a saúde e o desempenho acadêmico.

\subsection{Estacionamento de pergolados com trepadeiras}

  Como um dos objetivos propostos no projeto foi relacionar os problemas enfrentados na Faculdade do Gama com a agricultura urbana,
  analisamos o que poderia ser feito com uma das maiores dificuldades que é encontrada no campus, o estacionamento. Hoje na faculdade,
  há uma estrutura precária para os carros estacionarem e um alto nível de insegurança no local, e por conta desses fatores uma solução
  que não estaria somente visando uma melhor infraestrutura a comunidade acadêmica, serviria também como ocupação de uma área ociosa.

  A construção de estacionamentos de pergolados com trepadeiras seria a solução do problema, sendo que delimitaria as vagas e estaria
  levando maior segurança aos usuários do local, pois conforme uma pesquisa, foi analisado que, locais que antes eram ociosos/abandonados
  e passaram a ser ocupados, foram considerados mais seguros e em alguns lugares diminui o número de vandalismo.

\subsection{Telhado verde}

  O telhado verde é composto por camadas que de cima para baixo incluem: a vegetação, substrato, filtro de tecido de drenagem, camadas de
  retenção de água, camada de proteção da raiz, isolamentos, impermeabilização e um terraço, sendo algumas delas opcionais \cite{nutricao11}.

  Por se tratar de um projeto  que consiste no cultivo de plantas nos telhado dos prédio do campus, foi pensado na ideia de automatizar o
  sistema de irrigação e monitoramento das plantas, evitando assim que pessoas tenham que subir nos telhados para cuidar das plantas e
  futuros acidentes.

  A utilização dos telhados verdes no ambiente construído se dá, principalmente, pela constatação dos seus múltiplos benefícios, alguns
  dos quais poderiam ser aproveitados pela Faculdade do Gama da Universidade de Brasília  (UnB- FGA), tais como:

\textbf{Superaquecimento}: Na FGA ocorre calor intenso em determinadas épocas do ano, e a instalação do telhado verde ajudaria a
  amenizar este problema através do processo de evapotranspiração \cite{nutricao11}. Este processo, consistido pela evaporação e pela transpiração, é
  permitido pela capacidade do telhado de reter água da chuva, e ajuda a proporcionar eficácia no controle de drenagem, contribuindo para
  prevenção de enchentes.

  \textbf{A melhoria da qualidade da água}: A vegetação e o substrato funcionam como filtros e o excesso de água que a vegetação não
  absorve pode ser direcionada a uma cisterna e posteriormente ser aproveitada \cite{nutricao10}. O objetivo é instituir medidas que induzam a conservação,
  uso racional e utilização de fontes alternativas para captação de água nas novas edificações, bem como a conscientização dos usuários
  sobre a import\^{a}ncia da conservação da água. Além de outros como: A redução da ilha-de-calor urbano, a melhoria da qualidade do ar \cite{nutricao11},
  a formação de novos habitats reforçando o ecossistema, as razões estéticas, a recuperação do espaço de zoneamento, a conservação de
  energia na redução do uso de aparelhos direcionados a aquecer e resfriar ambientes, o benefício acústico com a redução de ruídos
  externos devido a sua espessura e característica da sua vegetação e substrato, o benefício da durabilidade e a redução da água de chuva
  escoada \cite{nutricao10}.

  \textbf{Reaproveitamento da água}: O teto verde funciona como um filtro para a água da chuva, e o excesso que ele não é capaz de
  absorver pode ser redirecionado por uma cisterna, e eventualmente reaproveitado. Na FGA, em períodos de chuva, ocorre o alagamento de
  muitas áreas da instituição e esta característica do teto verde ajudaria a minimizar este problema.
