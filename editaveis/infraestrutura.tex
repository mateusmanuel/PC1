\section{Infraestrutura}

\subsection{Áreas Potenciais}

  Grande parte da área destinada a FGA é composta por uma área denominada campo de murundu. Sobre o campo de murundu podemos elencar as
  seguintes características:

  \begin{itemize}
    \item Tipo de microrrelevo em forma de pequenas elevações, geralmente arredondados, com altura entre 0,1 a 1,5m e dimetro de até 20m;
    \item Temporariamente inundável nas partes mais baixas durante o período chuvoso;
    \item Formado em solos hidromórficos com deficiência em drenagem, contendo comumente no perfil concreções ferruginosas;
    \item É comum a presença do lençol freático à superfície ou próximo a superfície, devido a existência de uma camada de material
      impermeável, que junto com a baixa declividade, impede ou torna deficiente a drenagem nas áreas de murundus;
    \item Apresenta grande import\^{a}ncia ecológica por controlar o fluxo de água, a deposição de nutrientes, a conservação de água de
      superfície e a biodiversidade;
    \item Por possuir estas características e ser uma área de preservação ambiental, ela não foi considerada como área potencial
      para utilização no projeto.
  \end{itemize}

  A figura da demarcação das áreas potenciais se encontra no anexo \ref{areas-potenciais}.

  As áreas de 1 a 11 são espaços ocioso, porém os espaços ociosos utilizáveis para o projeto englobam as áreas de 2 até 11.

  \begin{description}
    \item[Área 1 $-$ Campo vazio] \
      \begin{itemize}
        \item Área com vegetação predominante do cerrado, com grande import\^{a}ncia ambiental e com import\^{a}ncia para o projeto de
          expansão da FGA.
      \end{itemize}
    \item[Área 2 $-$ Teto do UED] \
      \begin{itemize}
        \item É uma área com grande espaço disponível, onde inúmeras técnicas presentes em diversas construções ao redor do mundo podem
          ser utilizadas para ocupá-lo.
      \end{itemize}
    \item[Área 3 $-$ Espaço destinado ao estacionamento] \
      \begin{itemize}
        \item Esta é a maior área disponível para o plantio com fácil acesso no terreno da faculdade. Neste local podem ser construídas
          estruturas para o plantio que não afetem o piso local e que tragam algum benefício, como por exemplo sombra para os veículos
          estacionados.
      \end{itemize}
    \item[Área 4 $-$ Espaço entre o MESP e a quadra de esportes] \
      \begin{itemize}
        \item Espaço com grande potencial para utilização no projeto, por ser de fácil acesso da comunidade acadêmica e com uma área
          considerável disponível. Sendo assim um bom espaço apesar de apresentar alguns riscos provenientes da quadra de esportes localizada logo ao lado.
      \end{itemize}
    \item[Área 5 $-$ Canteiros ociosos dentro do MESP] \
      \begin{itemize}
        \item Pequenas áreas, mas que podem ser utilizadas para o plantio que pode contribuir na alimentação frequente que ocorre no local, por exemplo, podendo plantar pimenta, considerando o grande consumo dela nas refeições oferecidas pelo restaurante universitário.
      \end{itemize}
    \item[Área 6 $-$ Espaço entre o MESP e o UAC] \
      \begin{itemize}
        \item Área com grande espaço disponível e de fácil acesso por meio dos alunos. Podendo assim ser um local que acabe se tornando uma área de
          convivência dos alunos, pois este possibilita o plantio de árvores frutíferas de médio porte.
      \end{itemize}
    \item[Área 7 $-$ Teto do UAC] \
      \begin{itemize}
        \item É uma área com grande espaço disponível, onde inúmeras técnicas presentes em diversas construções ao redor do mundo podem
          ser utilizadas para ocupá-lo.
      \end{itemize}
    \item[Área 8 $-$ Espaço vazio embaixo do auditório] \
      \begin{itemize}
        \item Pequena área disponível, podendo assim ter plantas frutíferas de pequeno porte.Nesta área será necessário uma
          iluminação artificial para que determinadas plantas possam crescer.
      \end{itemize}
    \item[Área 9 $-$ Espaço entre os laboratórios de computação e o auditório] \
      \begin{itemize}
        \item Local com uma pequena área disponível, mas de fácil acesso por parte dos estudantesm, onde atualmente é utilizada como área de
          convivência entre os alunos.
      \end{itemize}
    \item[Área 10 $-$ Praça de convivência] \
      \begin{itemize}
        \item Área com um grande espaço disponível e já utilizada para a convivência entre os alunos. Nesta área existe a possibilidade de
          agregar mais valor por meio do cultivo de árvores frutíferas, trazendo mais sombras ao local e frutas em determinadas épocas do
          ano.
      \end{itemize}
    \item[Área 11 $-$ Canteiros ociosos dentro do UAC e UED] \
      \begin{itemize}
        \item Essas áreas sofrem com o despejo de resíduos por parte da equipe de limpeza, porém são áreas com grande potencial por serem de
          fácil acesso.
      \end{itemize}
  \end{description}

\subsection{Pesquisas de Viabilidade}

  A figura da distribuição dos tipos de solo no Distrito Federal pode ser encontrado no anexo \ref{tipos-solo}.

  \textbf{Tipo de solo}: Latossolo Vermelho

  \textbf{Análise do solo}: O latossolo são solos altamente relacionados à intemperização e lixiviação intensas, além disso são conhecido
  por serem solos profundos. “Os latossolo vermelho-escuro de forma dominante, tendem a ocupar áreas de topografia plana ou suavemente
  ondulada, como aquelas dos amplos chapadões do Brasil Central” \cite{infra2}. Segundo a Embrapa, são solos com alta permeabilidade à água, podendo
  ser trabalhados em grande amplitude de umidade. “Os latossolos são muito intemperizados, com pequena reserva de nutrientes para as
  plantas, representados normalmente por sua baixa a média capacidade de troca de cátions. Mais de 95\% dos latossolos são distróficos e
  ácidos, com pH entre 4,0 e 5,5 e teores de fósforo disponível extremamente baixos, quase sempre inferiores a 1 mg/dm cubico. Em geral, são
  solos com grandes problemas de fertilidade” \cite{infra3}.

  \textbf{Possibilidades e impossibilidades}: O solo é caracterizado por uma alta acidez, isso o torna pouco fértil em condições naturais.
  É necessário que haja uma correção nessa característica para aumentar o rendimento produtivo das culturas, para isso pode-se fazer uma
  correção com a aplicação de calcário. Posteriormente é feita a análise da deficiencia de fósforo. Para se adequar à plantação desejada,
  é feita a adequação do solo com adubo.

\subsection{Pré-Solução}

  Tipos de agricultura possíveis nas áreas em potenciais:

  \begin{description}
    \item[1. Tetos do UED e UAC] \

      Uma solução possível para essas áreas seria transformá-las em telhados verdes, que é uma solução arquitet\^{o}nica que consiste na
      aplicação de uma camada vegetal sobre uma base impermeável.

      Além de uma solução estética os telhados verdes são uma alternativa viável para a gestão de águas pluviais em áreas urbanas, pois
      retardam a drenagem pluvial, mitigando assim problemas com enchentes e saturação das galerias pluviais. São ainda uma ótima solução
      termoacústica, atuando como isolante evitando a transferência de calor (dificultando a formação de ilhas de calor), frio e ruído para
      o interior da edificação, desta forma minimizam gastos energéticos com aquecimento e refrigeração, constituindo se numa solução para
      a economia de energia.

      No que se refere a infraestrutura deve-se ter especial atenção à impermeabilização da base onde será implantando o telhado verde,
      assim como, seguir as diretrizes prevista na NBR15352 e NBR9952, atentar-se ainda para as cargas aplicadas pelo telhado verde sobre
      a estrutura que o receberá, observando-se um mínimo de 80 Kg por m2.

    \item[2. Área destinada a estacionamento] \

      Nessa área uma da alternativas encontradas foi a construção de pergolados com trepadeiras, pergolados são vigas dispostas paralelamente,
      apoiadas ou ligadas em outras vigas (estas, estruturais), ou muros. A função dessas vigas deixa de ser estrutural e passa a ser a de
      controle de luz, pois dessa forma, com o auxílio das trepadeiras essas estruturas acabam gerando sombras sobre os carros, diminuindo
      assim a exposição dos veículos aos eventos climáticos.

    \item[3. Canteiros ociosos dentro do MESP] \

      Nesses espaços é possível implementar mini-hortas, para que os próprios alunos pudessem usufruir de um bem saudável em sua
      alimentação, por exemplo: manjericão, cheiro-verde (salsa e cebolinha), coentro, tomilho, pimenta. Essas mini-hortas podem ser feitas
      com materiais reaproveitados, como pneus, garrafas PET, canos de PVC, baldes, latas, telhas, tambores, entre outras coisas.

      Outra alternativa seria a utilização das técnicas de hidroponia para a plantação dessas hortaliças e condimentos. O processo de
      hidroponia apresenta várias vantagens em relação às formas de cultivo tradicionais, como: crescimento mais rápido; maior produtividade;
      aumento da proteção contra doenças, pragas e insetos nas plantas; economia de água de até 70\% em comparação à agricultura tradicional;
      possibilidade de plantio fora de época e rápido retorno econ\^{o}mico; assim como menores riscos perante as adversidades climáticas.

    \item[4. Espaço entre o MESP e o UAC] \

      Por ser uma área com grande espaço disponível e de fácil acesso por meio dos alunos. O objetivo é tornar essa área em um local de
      convivência dos alunos, através do plantio de árvores frutíferas, gerando duas soluções para o espaço, local com sombra para que os
      alunos possam interagir, estudar e descansar, além da opção de frutas frescas disponíveis nas árvores.

    \item[5. Praça de convivência] \

      Assim como o item 4, o objetivo nessa área é o cultivo de árvores frutíferas que seria feito para sombrear a praça e refrescar, já
      que a presença de árvores aumentará a umidade.

    \item[6. Espaço vazio embaixo do auditório] \

      É uma área onde não é utilizada para nenhuma finalidade, deixando assim um espaço totalmente ocioso, e é um espaço onde o fluxo de
      aluno é muito grande, portanto ficaria fácil manter a plantação sobre constante observação. O problema que ocorreria para
      a utilização deste espaço é pouca incidência de raio solar, entretanto segundo o autor KOSAL, já existe a utilização de uma nova
      técnica de plantio conhecida como indoor, ou seja, é a plantação dentro de estrutura onde não há incidência de raios solares. Essa
      nova técnica consiste em utilizar l\^{a}mpadas de LED para poder substituir a luz solar.

  \end{description}
