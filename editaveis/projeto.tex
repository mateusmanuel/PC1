\section{Plano de gerenciamento de projeto}

\subsection{Objetivo}

  O Plano de Gerenciamento de Projeto visa integrar as partes do projeto e estabelecer como será garantida essa integração, visando o
  sucesso do projeto.

  Para isso, serão especificadas as ferramentas que serão utilizadas para a realização da integração do projeto e quais políticas de
  controle de mudança serão aplicadas.

\subsection{Controle de mudanças}

  O controle de mudanças será feito de forma integrada, permitindo o questionamento e acompanhamento do projeto por todas as equipes.
  Sendo assim, a responsabilidade é compartilhada e não pode ser restringida a uma única equipe. Para garantir isso, todas as mudanças
  devem ser comunicadas e rastreadas através das ferramentas descritas na seção Ferramentas.

\subsection{Controle de riscos}

  O plano de gerenciamento de risco se encontrar no anexo \ref{risco}.

\subsection{Tomada de decisão do projeto}

  As decisões de projeto, por ser uma metodologia ágil, serão tomadas tanto pela gerência do projeto como por todos os outros membros da
  equipe, entretanto para que as decisões sejam aceitas todos os membros da equipe, incluindo a gerência, devem estar de acordo com a
  decisão.

\subsection{Historico de alterações}

  Todos os artefatos devem possuir controle de versão que permita identificar quais alterações ocorreram no artefato, quem foi o autor da
  alteração, e qual foi a data da alteração. Também deve ser possível desfazer as alterações caso seja identificado qualquer problema.

  Para garantir esse compromisso, as ferramentas utilizadas para a documentação do projeto dão suporte ao controle de versão, que no caso
  é o GIT.

\subsection{Ferramentas}

  \begin{table}[!htb]
    \centering
    \begin{tabular}{p{3cm}p{10cm}}
      \toprule
        \textbf{Ferramenta} & \textbf{Proposito} \\
      \midrule
        Telegram      & Integrar as equipes do projeto para saber, entender e discutir o que está acontecendo em cada equipe, estabelecendo
                        uma comunicação constante entre elas.                 \\ \midrule
        Github        & Integrar os pequenos incrementos do projeto realizados individualmente por cada equipe e permitir o acompanhamento
                        do progresso de desenvolvimento do produto final.           \\ \midrule
        Trello        & Utilizando a estratégia de Kanban o trello permite o acompanhamento das tarefas que serão feitas em cada ponto de
                        controle, o que chamamos na metodologia ágil SCRUM de product backlog, o acompanhamento das tarefas que serão
                        entregues semanalmente, que na metodologia ágil chamamos de sprint semanal, e as tarefas já concluídas. Através
                        do trello conseguiremos fazer a retrospectiva da sprint analisando os pontos fortes e fracos da sprint e sua revisão
                        que são as revistas dos artefatos criados na sprint.          \\ \midrule
        Google Driver & Permitir o acompanhamento da produção dos artefatos em tempo real e com a participação de todos em paralelo,
                        possibilitando correções antes do documento ser publicado.  \\
      \bottomrule
    \end{tabular}
    \caption{Ferramentas de gerenciamento de projeto}
  \end{table}

